\documentclass[12pt,a4paper,titlepage]{article}
%\usepackage[doublespacing]{setspace}
\usepackage[utf8]{inputenc}         %This is used for ASKII digits.
\usepackage{amsmath, amssymb}       %This is for mathematic statements,check"amath.colorado.edu/documentation/LaTex/Symbols.pdf"
\usepackage{booktabs}
\usepackage{caption2}
\usepackage{epstopdf}
\usepackage{amsfonts,mathrsfs}      %fonts~~
\usepackage{graphicx}               %This is for graph inserting.
\usepackage{paralist}               %Give me compact lists!!!!! 'compactitem' 'compactenum' 'compactdesc'
%\usepackage{bm}                     %various kinds of bold texts.
\usepackage[top =2.54cm, bottom =2.54cm, left =3.18cm, right =3.18 cm]{geometry}
\usepackage{indentfirst}            %Indent the first letter
\usepackage{fancyhdr}               %Below is the head of the article.
\pagestyle{fancy}
\usepackage{lastpage}
\lhead{Team \# 33131}
\rhead{Page \thepage{} of \pageref{LastPage}}
\cfoot{}
%\numberwithin{equation}{section}    %Numbering of the equations will be based on sections.
%\usepackage{amsthm}
%\theoremstyle{definition}
%\newtheorem{def}{Definition}  %Below set the theorem system.
%\theoremstyle{plain}
%\newtheorem{thm}[law]{Thm}
%\theoremstyle{remark}
%\newtheorem*{remark}{Remark}
\newcommand{\boldit}[1]{\textbf{\textit{#1}}}
\renewcommand{\captionlabelfont}{\small \bfseries \rmfamily}
\usepackage{float}
\usepackage{paralist}
\usepackage{url}

\begin{document}

\title{Human Capital Model} \date{\today{}}
\maketitle

\tableofcontents

\newpage

\section{Introduction}
\label{sec:introduction}
\begin{displaymath}
Q=\iiint\limits_{\Omega}\rho C_{w}(u_f - u_i)dx dy dz
\end{displaymath}

\begin{displaymath}
\begin{aligned}
\int_{t_1}^{t_2} \iint\limits_{\Gamma}k\frac{\partial u}{\partial n}dsdt & =\iiint\limits_{\Omega}\rho C_{w}(u_f - u_i)dx dy dz\\
& =\int_{t_1}^{t_2}\iiint\limits_{\Omega}[\frac{\partial}{\partial x}(k\frac{\partial u}{\partial x})+\frac{\partial}{\partial y}(k\frac{\partial u}{\partial y})+\frac{\partial}{\partial z}(k\frac{\partial u}{\partial z})]dxdydx\\
& = \iiint\limits_{\Omega}\rho C_{w}(\int_{t_1}^{t_2}\frac{\partial u}{\partial t}dt)dxdydz
\end{aligned}
\end{displaymath}

\begin{displaymath}
\rho C_{w}\frac{\partial u}{\partial t}=k\bigtriangleup u+\frac{C_w V_h(u_h-u)}{V}-\frac{C_v g}{V}
\end{displaymath}


\begin{displaymath}
Q=\iiint\limits_{\Omega}\rho C_{w}(u_f - u_i)dx dy dz
\end{displaymath}

\begin{displaymath}
\begin{aligned}
\int_{t_1}^{t_2} \iint\limits_{\Gamma}k\frac{\partial u}{\partial n}dsdt & =\iiint\limits_{\Omega}\rho C_{w}(u_f - u_i)dx dy dz\\
& =\int_{t_1}^{t_2}\iiint\limits_{\Omega}[\frac{\partial}{\partial x}(k\frac{\partial u}{\partial x})+\frac{\partial}{\partial y}(k\frac{\partial u}{\partial y})+\frac{\partial}{\partial z}(k\frac{\partial u}{\partial z})]dxdydx\\
& = \iiint\limits_{\Omega}\rho C_{w}(\int_{t_1}^{t_2}\frac{\partial u}{\partial t}dt)dxdydz
\end{aligned}
\end{displaymath}

\begin{displaymath}
\rho C_{w}\frac{\partial u}{\partial t}=k\bigtriangleup u+\frac{C_w V_h(u_h-u)}{V}-\frac{C_v g}{V}
\end{displaymath}

\subsection{Restatement of the Problem}
\label{sec:restatement-of-the-problem}
We are required to build a mathematical model to solve the problem of heating water in bathtub. Thus we have some subproblems:
\begin{itemize}
\item Build a basic model that demonstrate the change of the temperature in the bathtub in space and time without other intervention.
\item Figure out the influence of parameters such as the shape of the bathtub or the motions made by the person or so.
\item Propose a strategy to keep the temperature as even as possible.
\end{itemize}
In the first step, we build the simplest model. The shape of the bathtub is a cuboid, and the person will stay still. In the second step, the person move slowly and the shape of the bathtub changes. In the third step, we develop the conclusion of the  strategy.


\section{Assumptions and Justifications}
\label{sec:assumptions-and-justifications}

\begin{itemize}
\item \textbf{assumption 1} If an employee has probability to
  promote, he won't churn.

The possibility of the unforeseen accidents, which could force an
employee to leave his position,is neglected.Human nature, an employee
will stay at his position to chase for higher level.

\item \textbf{assumption 2} For a vacancy,if there exists an
  employee measures up to it already,ICM won't recruit for it.

Since recruiting good people is difficult, time consuming and
expensive according to issue 5,it is wasteful to recruit for a
position if an employee can promote to it.

\item \textbf{assumption 3} Demotion won't occur.

\item \textbf{assumption 4} Administrative clerk won't promote or be
  transferred.

\item \textbf{assumption 5} For the promotion probability and
  organization change, the other factors effects the churn probability
  is invariable.

Though churn derives from varieties of reasons and they are actually
lacking of known conditions and data to estimate them, we have to
regard it as stable in our model.

\item \textbf{assumption 6} Each division or office have at least one
  middle manager or senior manager.

\end{itemize}

% the main section of the paper
\section{Water in the Bathtub Model}
\label{sec:human-capital-model}
Definitions of symbols employed in this paper are listed in
\textbf{Table 1}.
\begin{table}
\begin{tabular}{l|l}
  Constant & Description \\
  \hline
  $k$            &Thermal conductivity of water \\
  $\rho$         &Density of water \\
  ${\gamma}_1$   &Heat exchange coefficient between water and air \\
  ${\gamma}_2$   &Heat exchange coefficient between water and person \\
  ${\gamma}_3$   &Heat exchange coefficient between water and bathtub \\
  $C_w$          &Specific heat capacity of water \\
  $V$            &Total volume of water \\
  $x_s$          &Humidity ratio in saturated air \\
  $x_0$          &Initial humidity ratio in the air \\
  $A$            &Square of water surface \\
  $V_a$          &Volume of surrounding air \\
  $C_v$          &Evaporation heat of water \\
  $u_2$          &Temperature of person's surface \\

  Variable & Description \\
  \hline
  $u$            &Temperature of water \\
  $u_1$          &Temperature of air \\
  $u_3$          &Temperature of bathtub \\
\end{tabular}
\end{table}

\subsection{Model Overview}
\label{sec:model-overview}

Partial differential equations are widely used in natural science, engineering and economical management.
It is natural to use partial differential equation to solve this physical problem.

In our model, five parts accounts for heat transfer of water: heat exchange between water and air, heat
exchange between water and the person, heat exchange between water and the bathtub, heat of hot water from the
faucet and heat of water flow outside. According to thermology, we conduct the equation
of water temperature in the bathtub. The initial-boundary value condition could be conducted according to
Fourier theorem, vaporization equation[ ].

Since the equation is a relatively complex 3D partial differential equation, we use matlab to solve it.
We build a geometry as the bathtub. By finite element method, the geometry discrete into several parts.
The solution of each part is considered to be a relatively precise value solution.



\subsection{Heat Equation}
\label{sec:heat equation}

In this part, we use physical theorem to conduct the heat conduction equation
in the bathtub.

Figure 1 shows thermal analysis of water in bathtub.

First,we assume that the heat change of all water in a tiny time $\Delta t$ from $t_1$ to $t_2$ is $\Delta Q$.
As shown in Figure 1, water in the bathtub have heat exchange with air, the person, and the bathtub.
In addition, heat changes when the hot water flow in and the excess water flow out.
Thus $\Delta Q$ is divided into five parts, i.e.
\begin{equation}
 \Delta Q=\Delta Q_1+\Delta Q_2-\Delta Q_3-\Delta Q_4
\end{equation}
in which $\Delta Q_1$ is heat transferred from air, person and bathtub,
$\Delta Q_2$ is the heat of the hot water from the faucet in $\Delta t$,
$\Delta Q_3$ is the heat of the water flow away in $\Delta t$,
$\Delta Q_4$ is the heat loss of vaporization.

Totally, internal energy of water changes while heat changes.
Then the temperature changes with
\begin{equation}
 Q=\iiint\limits_{\Omega}\rho C_{w}(u_f - u_i)dx dy dz
\end{equation}
, in which $\Omega$ is the region of water, $\rho$ is the density of water, $u_f$ is the final temperature when t=$t_2$, $u_i$ is the initial temperature when t=$t_1$.

Considering the right side of equation 1.
Fourier theorem indicates that,
\begin{equation}
 dQ=-k\frac{\vartheta u}{\vartheta n}dSdt
\end{equation}
holds for the heat flow through surface with square of dS, as shown in Figure 2.
In the equation, k is thermal conductivity, $\frac{\vartheta u}{\vartheta n}$ is the derivative of u along the normal line. Thus, after integration of dQ by time and space
\begin{equation}
 \Delta Q_1=int_{t_1}^{t_2}[{int_{\Gamma}}k\frac{\vartheta u}{\vartheta n}dS]dt
\end{equation}
in which $\Gamma$ is the boundary of $\Omega$.
The heat flow into the water by thermal conduction could be calculated by the equation.

$\Delta Q_2$ is determined by the internal energy of hot water from the faucet in $\Delta t$.
And $\Delta Q_2$ can be calculated by
$\Delta Q_2={\Delta m}C{u_h}$.
Since the hot water is constant,
$\Delta m=v{\Delta t}$.

It is obvious that the excess water is as the same value of hot water in any time period.
Thus $\Delta Q_3$ can be calculated by
$\Delta Q_3={\Delta m}{C_w}u$.
Also, $\Delta m=v{\Delta t}$.
$\Delta v$ is the weight of hot water each second, $\Delta m$ is the weight of hot water in $\Delta t$.

According to thermal theorem, heat loss in vaporization is proportional to the weight of vapor water.
$\Delta Q_4={C_v}\Delta {m_v}$.
$\Delta {m_v}$ is the weight of evaporation water in $\Delta t$, $\Delta {C_v}$ is evaporation heat of water.

Since water could be considered as manifold, according to Green formula,
\begin{equation}
% \int_{t_1}^{t_2} \iint\limits_{\Gamma}k\frac{\partial u}{\partial n}dsdt & =\iiint\limits_{\Omega}\rho C_{w}(u_f - u_i)dx dy dz\\
%& =\int_{t_1}^{t_2}\iiint\limits_{\Omega}[\frac{\partial}{\partial x}(k\frac{\partial u}{\partial x})+\frac{\partial}{\partial y}(k\frac{\partial u}{\partial y})+\frac{\partial}{\partial z}(k\frac{\partial u}{\partial z})]dxdydx\\
%& = \iiint\limits_{\Omega}\rho C_{w}(\int_{t_1}^{t_2}\frac{\partial u}{\partial t}dt)dxdydz%
\end{equation}.

Combining the equations above,
\begin{equation}
% \rho C_{w}\frac{\partial u}{\partial t}=k\bigtriangleup u+\frac{C_w V_h(u_h-u)}{V}-\frac{C_v g}{V}%
\end{equation}.
In this heat equation, $u_h$ and $v_h$ are two variables that control the strategy.

Now discuss g, evaporation rate per square.
As Figure 3 shows, water evaporate into air and air humidity increases, until it saturate. According to empirical formula in [], we know that
\begin{equation}
 g=\frac{(25+19v)({x_s}-x)}{3600}
\end{equation}
in which v is the velocity of air, ${x_s}$ is humidity ratio in saturated air, $x$ is humidity ratio in air.
Considering the humidity ratio in the air, it indicates
\begin{equation}
 x=x_0+\int \frac{Ag}{\rho V_a}dt
\end{equation}
in which, $x_0$ is the initial humidity ratio in the air, $\rho$ is density of water, $V_a$ is the total air volume of surrounding air.

Combing the two equations and differentiating, we get a ordinary partial equation
\begin{equation}
 \frac{3600}{25+19v} \frac{dg}{dt}=\frac{A}{\rho {V_a}}
\end{equation}
Solving the equation, we get the solution
\begin{equation}
 g=\frac{(25+19v)({x_s}-{x_0})}{3600}e^{-\lambda t}, \lambda =\frac{A(25+19v)}{3600\rho {V_a}}
\end{equation}


\subsection{Initial-Boundary Value Condition}
\label{initial-boundary value condition}

Boundary value condition is quite essential in partial differential equations. In our model, water is not a regular research object.
Thus boundary value condition is complex.

The boundary of water consists of four parts: water-air boundary, water-person boundary, water-bathtub boundary, and the square for faucet. We denote them as
${\Gamma}_1$, ${\Gamma}_2$, ${\Gamma}_3$, ${\Gamma}_4$ respectively.

Since person is temperature-constant, the boundary value condition of ${\Gamma}_1$ and ${\Gamma}_3$ could be considered
as the Dirichlet value condition. That is
\begin{equation}
 u=u_i, (x,y,z)\in {\Gamma}_i, i=2.4
\end{equation}
In the equation, $u_2$ is the surface temperature of person. We assume it is constant. $u_4$ is the temperature of hot water.

The boundary value condition of ${\Gamma}_1$ and ${\Gamma}_3$ could be considered as the third boundary value condition approximately. That is
\begin{equation}
 k\frac{\vartheta u}{\vartheta n}+{{\gamma}_i}u={{\gamma}_i}{u_i}, (x,y,z)\in {\Gamma}_i
\end{equation}
In the equation, i=1,3, and $u_i$ is the temperature of contact material,
${\gamma}_i$ is the heat exchange coefficient of ${\Gamma}_i$.

In the equation above, $u_1$, $u_3$ are temperatures of air and bathtub, they are time-dependant functions. It takes several steps to calculate them.

Consider the temperature of air, $u_1$. Heat change of air consists of two ways, heat conduction and evaporation.

In our model, we assume the temperature of water is equilibrium throughout the bathtub initially.

\subsection{PDE Solution}
\label{sec:PDE solution}
Since our equations are complex and the region is quite irregular, it's impossible to get a regular solution. Finite element method is normally adopted to deal with this problem.

We use Partial Differential Equation Toolbox in matlab to solve our equations. Partial Differential Equation Toolbox makes use of finite element method to discrete irregular geometry and calculate a relatively precise solution. get solution of temperature corresponding to different $u_h$ and $v_h$. By comparing temperature distribution, we choose proper parameters as the best strategy.

First, we build a 3D geometry model to simulate a proper-size bathtub, as shown in Figure 2. Matlab divided the geometry into several elements.

Then we input the heat equation and initial-boundary value conditions. Boundary value conditions are different in three parts.

\section{Results and Analysis}
\label{sec:performance-and-analysis}

Our strategy focus on two variables, $v_h$ and $u_h$. Several pairs of data are tested to find the best strategy.
\begin{center}
\begin{tabular}{|c|c|c|}
\hline
      &$v_h$  &$u_h$        \\ \hline
 1    &             & 5              \\ \hline
 2    &             & 1              \\ \hline
 3    &             & 3              \\ \hline
 4    &             & 4              \\ \hline
\end{tabular}
\end{center}


\section{Sensitivity Analysis}
\label{sec:sensitivity-analysis}

\subsection{Parameters of Bathtub}
\label{sec:parameters of bathtub}
\begin{itemize}
\item We change the shape of the bathtub to sphere and elliptical with the same strategy. The temperature graphic shows that sphere is the most proper shape for keeping water temperature. Square performs the worst on keeping temperature. However, temperature of square bathtub is more evenly than others.
\item We change the volume of the bathtub without changing the proportion. The result as following Table [] shows that with the same $v_h$, $u_h$ increases as volume increases; with the same $u_h$, $v_h$ increased as volume increases.
\end{itemize}

\subsection{Parameters of Person}
\label{sec:parameters of person}
\begin{itemize}
\item
\item
\item
\end{itemize}

\subsection{Effects of Motion}
\label{sec:effects of motion}
When the person in the bathtub make motions, the heat exchange coefficient between person and water ${\gamma}_2$ increases. That means heat exchange between water and person increases. 
 
In another aspect, the air above flow, so that evaporation heat of water increases.

\subsection{Effects of Bubble Bath Additive}
\label{sec:effects of bubble bath additive}

\section{Strengths and Weaknesses}
\label{sec:strengths-and-weaknesses}

\subsection*{Strengths}
\label{sec:strengths}

\begin{itemize}
\item Our model make good use of thermology knowledge so it can reflect the situation in the real world realistically.
\item Our model build a 3D geometry to simulate the research object so 
\item The network we built include both microcosmic part and
  macrocosmic part, and they react to each other.
\item Our model proves the effect of time.
\end{itemize}

\subsection*{Weaknesses}
\label{sec:weaknesses}

\begin{itemize}
\item We assume that the water is still. In fact, streams flow and fluid dynamic is better taken into consideration. The actual situation is much complicated.
\item The model of the person should be more realistic. In fact, model of human body includes hands, legs or other parts, and the surface of human skin isn't even.
\item The condensation of water should be taken into consideration. Water that evaporate become water vapour, and some will condense into water on the bathtub.
\end{itemize}


% the reference
\begin{thebibliography}{99}

\bibitem{}

\end{thebibliography}

\label{LastPage}

\end{document}

%%% Local Variables:
%%% mode: latex
%%% TeX-master: t
%%% End:
